\documentclass{exam}
\usepackage[overload]{exam-randomizechoices}
\usepackage{amsmath}
\usepackage{amsfonts}
\usepackage{tikz}
\usetikzlibrary{calc}

\begin{document}
\begin{questions}
        
    \question
    Which statement best describes the study of cooperation in Game Theory?

    \begin{checkboxes}
        \choice The study of cooperation in game theory examines how and why players collaborate to achieve mutually beneficial outcomes, even when individual incentives may favor non-cooperation.
        \choice The study of cooperation in game theory focuses on the design of games where cooperation emerges naturally as the dominant strategy due to carefully aligned incentives.            
        \choice The study of cooperation in game theory is concerned with analyzing zero-sum games where collaboration between players is impossible due to conflicting interests.
        \choice The study of cooperation in game theory involves determining the optimal punishment strategies to prevent players from forming alliances or coalitions.
    \end{checkboxes}

    \question
    Which of the following games is a Prisoner's Dilemma?

    \begin{checkboxes}
        \choice \(R=3, S=0, T=5, P=-1\)
        \choice \(R=1, S=0, T=500, P=1/2\)
        \choice \(R=4, S=0, T=5, P=0\)
        \choice \(R=-10, S=-3, T=-8, P=-4\)
    \end{checkboxes}

    \question What statement best describes the outcomes of Robert Axelrod's
    original tournaments?

    \begin{checkboxes}
        \choice In Robert Axelrod's original tournaments, the strategy ``Tit-for-Tat'' emerged as the most successful, demonstrating the importance of cooperation, reciprocity, and forgiveness in repeated games.
        \choice In Robert Axelrod's original tournaments, random strategies occasionally outperformed structured strategies, suggesting that unpredictability can be beneficial in repeated games.
        \choice In Robert Axelrod's original tournaments, aggressive strategies were effective against cooperative strategies but ultimately failed in the long run, highlighting the risks of defection in repeated games.
        \choice In Robert Axelrod's original tournaments, all strategies had similar long-term outcomes, suggesting that no single approach consistently dominates in repeated games.
    \end{checkboxes}

\end{questions}
\end{document}
