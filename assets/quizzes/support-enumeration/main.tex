\documentclass{exam}
\usepackage[overload]{exam-randomizechoices}
\usepackage{amsmath}
\usepackage{amsfonts}

\begin{document}
\begin{questions}
        
    \question
    Which statement best describes the support enumeration algorithm?

    \begin{checkboxes}
\choice The support enumeration algorithm is a method for finding mixed strategy Nash equilibria by iterating over all possible combinations of supports for players and checking feasibility conditions.
\choice The support enumeration algorithm is used to find pure strategy Nash equilibria by iterating through all possible combinations of strategies and checking for mutual best responses.
\choice The support enumeration algorithm identifies Nash equilibria by simulating repeated gameplay and observing which strategies survive over time.
\choice The support enumeration algorithm calculates equilibria by directly solving a matrix of payoffs without considering the players’ strategy supports.
    \end{checkboxes}


\question
    For a non-degenerate game \((A, B)\in{\left(\mathbb{R}^{4 \times 3}\right)}^2\) how many pairs of supports
    will be checked using the support enumeration algorithm?

    \begin{checkboxes}
        \choice \(14\)
        \choice \(7\)
        \choice \(98\)
        \choice \(105\)
    \end{checkboxes}

    \question Which following pair of strategies is a Nash equilibrium for the
    following game:
     \[
     A=\begin{pmatrix}4 & 3\\ 0& 5\end{pmatrix}
     \qquad
     B=\begin{pmatrix}5 & 2\\ 3& 8\end{pmatrix}
 \]
    \begin{checkboxes}
        \choice
    \(
    \sigma_r=(5/8, 3/8)\qquad\sigma_c = (1/2, 1/2)
    \)
        \choice
    \(
    \sigma_r=(5/8, 3/8)\qquad\sigma_c = (1/3, 2/3)
    \)
        \choice
    \(
    \sigma_r=(1/4, 3/4)\qquad\sigma_c = (1/2, 1/2)
    \)
\end{checkboxes}

\end{questions}
\end{document}
