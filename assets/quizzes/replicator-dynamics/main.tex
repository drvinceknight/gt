\documentclass{exam}

\usepackage[overload]{exam-randomizechoices}
\usepackage{amsmath}
\usepackage{amsfonts}

\begin{document}
\begin{questions}
        
    \question
    Which statement best describes the replicator dynamics equation?

    \begin{checkboxes}
        \choice The replicator dynamics equation models the change in population proportions based on the difference between individual strategy payoffs and the average population payoff.
            \choice The replicator dynamics equation describes how the total population size remains constant while individual strategy proportions evolve based on fitness levels.
            \choice The replicator dynamics equation ensures that all strategies eventually converge to an equal proportion in the population.
            \choice The replicator dynamics equation predicts that the proportion of each strategy remains constant regardless of fitness differences.
    \end{checkboxes}

    \question What is a stable population \(x\) for the replicator dynamics
    equation with \(A=\begin{pmatrix}5 & 3\\ 1 & 4\end{pmatrix}\)

    \begin{checkboxes}
        \choice \((1/5, 4/5)\)
        \choice \((1/10, 9/10)\)
        \choice \((5/7, 1/7)\)
        \choice \((1/2, 1/2)\)
    \end{checkboxes}

    \question In the replicator-mutation dynamics equation which of the
    following matrices \(Q\) correspond to individuals of the 2nd type
    mutating to individuals of the 3rd type 20\% of the time and individuals of
    the 3rd type mutating to individuals of the 1st type 75\% of the time.
    \begin{checkboxes}
        \choice
    \(
        Q = \begin{pmatrix}1 & 0 & 0 & 0\\0 & 4/5 & 1/5 & 0\\ 3/4 & 0 & 0 & 1/4
        \\ 0 & 0 & 0 & 1\end{pmatrix}
    \)
        \choice
    \(
        Q = \begin{pmatrix}1 & 3/4 & 0 \\1/5 & 0 & 4 /5\\0 & 0 & 1\end{pmatrix}
    \)
        \choice
    \(
        Q = \begin{pmatrix}1 & 0 & 0\\0 & 4/5 & 1/5\\ 0 & 3/4 & 1/4 \end{pmatrix}
    \)
        \choice
    \(
        Q = \begin{pmatrix}1 & 0 & 0 & 0\\0 & 0 & 4/5 & 1/5\\ 0 & 3/4 & 0 & 1/4
        \\ 0 & 0 & 0 & 1\end{pmatrix}
    \)
\end{checkboxes}

\end{questions}
\end{document}
