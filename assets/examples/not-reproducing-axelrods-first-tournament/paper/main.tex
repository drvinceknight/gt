\documentclass[a4paper]{article}

% Set up page size
\usepackage[margin=1.5cm, includefoot, footskip=30pt]{geometry}

\usepackage{amsmath}
\usepackage{hyperref}
\usepackage{minted}
\usepackage{graphicx}

\title{Not reproducing Axelrod's original tournament}
\author{Vince Knight and the maintainers of the Axelrod library}
\date{2022--03--24}

\begin{document}

\maketitle

\begin{abstract}
The Iterated Prisoner's Dilemma is a model used to understand the emergence of
cooperation in complex situations. One of the first papers on this topic
presented the results of a computer tournament.

This paper presents the unsuccessful attempts made to reproduce these results.
As well as describing the general conclusions as it pertains to game theory a
discussion about reproducibility of scientific research will also be given.
\end{abstract}

\section{Introduction}

The Prisoner's Dilemma was first described in~\cite{flood1958some} and is
represented using the following matrix and constraints:

\begin{equation}
    A = \begin{pmatrix}
            R & S\\
            T & P\\
        \end{pmatrix}
        \qquad
    T > R > P > S
        \qquad
    2R > T + S
\end{equation}

The constraints ensure a social dilemma where rational actions lead to a Nash
equilibrium where both players obtain less than a coordinated outcome.
Choosing the first row or column are referred to as ``cooperating'' whereas
choosing the second row or column are referred to as ``defecting''.

The repeated version of the Prisoners Dilemma: the Iterated Prisoners Dilemma
has given rise to a large amount of research. A systematic review of this
research is available at~\cite{glynatsi2021bibliometric}. The initial pieces of
this body of work relate to the 2 computer tournaments run by Robert Axelrod in
the 1980s. These were originally reported in~\cite{axelrod1980effective,
axelrod1980more} but subsequently lead to~\cite{axelrod1981evolution} which has
over 45,500 citations.

The winning of both of the above tournaments was a strategy called
Tit For Tat: a strategy that starts by cooperating and then just mimics
the previous move of the opponent.

In the next section of this paper an attempt to replicate the work of Robert
Axelrod will be described.

\section{Results}

Previous work by the authors includes the design and continued maintenance
of the Python axelrod library~\cite{knight2016open}. This library is written following best
practices in software engineering, most notably that tests are written for all
code to ensure accurate behaviour. Secondly, this library is open source which
has allowed it to have a large number of contributors this not only adds to the
diversity of the code base (specifically to the large number of strategies
available) but also further ensures its correctness.

The descriptions of the strategies in Robert Axelrod's first
paper~\cite{axelrod1980effective} are
all that is available to be able to reproduce the work: the source code is not
available. Some of these are not precise, for example here is a quote from the
paper describing one of the strategies:

\begin{quote}
"This rule plays Tit for Tat except that it cooperates on the first four moves,
it defects on the last two moves, and every fifteen moves it checks to see
if the opponent seems to be playing randomly. This check uses a chi-squared
test of the other’s transition probabilities and also checks for alternating
moves of CD and DC."
\end{quote}

What is specifically meant by "alternating moves of CD and DC" is ambiguous.

Following careful investigation of the paper the strategies have been
implemented in the \mintinline{python}{axelrod} library and so the tournament
can be rerun.

Figure~\ref{fig:results_from_tournament} shows the result of a given run of the
tournament.

\begin{figure}[!htbp]
    \begin{center}
        \includegraphics[width=.3\textwidth]{../assets/results_from_tournament.pdf}
    \end{center}
    \caption{The reproduced results of Axelrod's first tournament.}
    \label{fig:results_from_tournament}
\end{figure}

This shows that the results are not actually reproduced. There are a few reasons
for which this could be the case:

\begin{enumerate}
    \item The implementation of the strategies do not correspond to the
        descriptions in~\cite{axelrod1980effective}.
    \item The stochastic effects.
    \item The descriptions in~\cite{axelrod1980effective} do not correspond to the results reported.
\end{enumerate}

The first reason is unlikely: a blog post was published at:
\begin{center}
    \url{https://vknight.org/blog/posts/reproducing-axelrods-first-tournament/}
\end{center}
due to the open nature of this research project some details were improved in
implementations.

To rule out the second reason the tournament was run over a large number of
repetitions. The best number of strategies who obtained the same rank was 11 and
this occurred less than 0.002 \textbf{percent} of the time.

\section{Conclusion}

This paper presents a study aiming to reproduce the work of Axelrod's first
tournament. To the best of the authors' knowledge it is not possible to
reproduce these results.

Whether or not the research is reproducible corresponds to an ongoing
conversation in modern scientific research~\cite{wilson2014best}. Perhaps a more
interesting questions is whether or not the overall conclusions remain the same.
One of the main conclusions from the victory of Tit for Tat was that cooperation
can emerge in complex situations.
Figure~\ref{fig:the-evolution-of-the-cooperation-rate} shows the overall
cooperation rate through the evolutionary process through replicator dynamics:
cooperation still emerges.

\begin{figure}[!htbp]
    \begin{center}
        \includegraphics[width=.3\textwidth]{../assets/the_evolution_of_the_cooperation_rate.pdf}
    \end{center}
    \caption{The evolution of the cooperation rate for the tournament of Figure~\ref{fig:results_from_tournament}}
    \label{fig:the_evolution_of_the_cooperation_rate}
\end{figure}

\bibliographystyle{plain}
\bibliography{bibliography.bib}

\end{document}
