
\documentclass[12pt,a4paper]{article}
\usepackage{amsmath,amssymb}
\usepackage{graphicx}
\usepackage{float}
\usepackage{tikz}
\usetikzlibrary{calc}
\usepackage[english]{babel}
\usepackage[utf8]{inputenc}
\usetikzlibrary{shapes, arrows, positioning}

\usepackage{amsmath,amssymb,mdframed}
\setcounter{page}{2}
\setlength{\oddsidemargin}{-0.25in}
\setlength{\textwidth}{6.5in}
\setlength{\topmargin}{-0.5in}
\setlength{\headsep}{1cm}
\setlength{\textheight}{9.0in}

\usepackage{environ}
\newif\ifshowsolutions

\NewEnviron{solution}{%
  \ifshowsolutions
  \hrule
  \vspace{.5cm}
    \par\noindent\textbf{Solution:} 

    \BODY
  \vspace{.5cm}
    \hrule
  \fi
}

% \showsolutionstrue

\makeatletter
\renewcommand{\@oddhead}{\hfill MA3600/example}
\renewcommand{\@oddfoot}{\hfil \arabic{page} \hfil}
\makeatother

\renewcommand{\labelenumi}{\arabic{enumi}}
\renewcommand{\labelenumii}{(\alph{enumii})}
\renewcommand{\labelenumiii}{(\roman{enumiii})}

\begin{document}
\begin{enumerate}

\renewcommand\labelenumi{\bfseries\theenumi.}

\item
    \begin{enumerate}
        \item Provide definitions for the following terms:
            \begin{itemize}
                \item Normal form game.

                    \begin{solution}
                        Book Work. \hfill{[1]}
                    \end{solution}

                \item Strictly dominated strategy.

                    \begin{solution}
                        Book Work. \hfill{[1]}
                    \end{solution}

                \item Weakly dominated strategy.

                    \begin{solution}
                        Book Work. \hfill{[1]}
                    \end{solution}

                \item Best response strategy.

                    \begin{solution}
                        Book Work. \hfill{[1]}
                    \end{solution}

                \item Mixed strategy Nash equilibrium.

                    \begin{solution}
                        Book Work. \hfill{[1]}
                    \end{solution}
            \end{itemize}
            ~\hfill{[5]}

        \item State and prove a theorem giving a condition for which a strategy
            of the row player
            is a best response to a given strategy of the column player.
            ~\hfill{[8]}

        \item Consider the following Normal Form Game defined by:

            \[
                M_r=
\begin{pmatrix}
4 & -2 \\
-1 & 3 \\
\end{pmatrix}
\qquad
            M_c=
\begin{pmatrix}
2 & -2 \\
-3 & 2 \\
\end{pmatrix}
\]

        State and justify which pairs of strategies are best responses to each
            other:
            \begin{enumerate}
                \item \(\sigma_r=(1, 0)\) and \(\sigma_c=(0, 1)\)
                \item \(\sigma_r=(1/5, 4/5)\) and \(\sigma_c=(0, 1)\)
                \item \(\sigma_r=(1/5, 4/5)\) and \(\sigma_c=(1/2, 1/2)\)
            \end{enumerate}
            ~\hfill{[9]}


        \item Using your answer to (c) or otherwise, find all Nash equilibria
            for the game.
            ~\hfill{[4]}
        
    \end{enumerate}

\newpage

\item
    Consider the donation game defined by:

    \[
        M_r = 
        \begin{pmatrix}    
            b + c & c\\
            b + 2 c    & 2 c
        \end{pmatrix}
        \qquad
        M_c = 
        \begin{pmatrix}    
            b + c & b + 2c\\
            c    & 2 c
        \end{pmatrix}
    \]


    \begin{enumerate}
        \item Show that if \(b>c>0\) then this game is a Prisoner's Dilemma.
            ~\hfill{[3]}
    \item Obtain all Nash equilibrium for this game assuming the constraints of
        (a).
            ~\hfill{[2]}

        \item Consider a Moran Process on this game. Obtain an expression for
            the fixation probability of \(i\) mutants: playing the first strategy in
            a population of with \(N\) as a function of \(b, c\) and \(N\).

            You may use the following expression for the fixation probability in the general two type Moran
            process:
            \[
\rho_i=\frac{1+\sum_{j=1}^{i-1}\prod_{k=1}^j\gamma_k}{1+\sum_{j=1}^{N-1}\prod_{k=1}^j\gamma_k}
\]

where:

            \[
\gamma_k = \frac{f_R(i)}{f_M(i)}
\]
            Where \(f_R(i)\) and \(f_M(i)\) is the fitness of a resident/mutant
            respectively in a population with \(i\) mutants.
            ~\hfill{[8]}
        \item Obtain the probability of a single mutant taking over for
            \(N\in\{2, 3, 4\}\).
            ~\hfill{[6]}
        \item For \(N=4\) consider the limit as \(b\to \infty\) and as \(b\to
            c\). Comment on the implications of these results.
            ~\hfill{[6]}
    \end{enumerate}

\newpage
\item

    \begin{enumerate}
        \item Define a characteristic function game \(G=(N,v)\).

        \hfill{[2]}

        \item Define the Shapley value.

        \hfill{[2]}

        \item Obtain the Shapley value for the following characteristic function games:

            \[
                v_1(c) = \begin{cases}
                    0,& \text{if }c=\emptyset\\
                    8,& \text{if }c=\{1\}\\
                    5,& \text{if } c=\{2\}\\
                    9,& \text{if } c=\{3\}\\
                    10,& \text{if } c=\{1,2\}\\
                    11,& \text{if } c=\{2,3\}\\
                    18,& \text{if } c=\{1,3\}\\
                    30,& \text{if } c=\{1,2,3\}\\
                \end{cases}
                \qquad
                v_2(c) = \begin{cases}
                    0,& \text{if }c=\emptyset\\
                    80,& \text{if }c=\{1\}\\
                    10,& \text{if } c=\{2\}\\
                    12,& \text{if } c=\{3\}\\
                    80,& \text{if } c=\{1,2\}\\
                    12,& \text{if } c=\{2,3\}\\
                    80,& \text{if } c=\{1,3\}\\
                    80,& \text{if } c=\{1,2,3\}\\
                \end{cases}
            \]

        \hfill{[8]}

        \item 
Given a game $G = (N, v)$, a payoff vector $\lambda$ satisfies the symmetry property if, for any pair of players $i, j$:

If $v(C \cup i) = v(C \cup j)$ for all coalitions $C \subseteq \Omega \setminus \{i, j\}$, then:

$$
\lambda_i = \lambda_j
$$

            Prove that the Shapley value is efficient.

        \hfill{[6]}

        \item The additivity property is:

        Given two games \(G_1 = (N, v_1)\) and \(G_2 = (N, v_2)\), define their sum \(G^+ = (N, v^+)\) by:

        \[
        v^+(C) = v_1(C) + v_2(C) \quad \text{for all } C \subseteq \Omega
        \]

        A payoff vector \(\lambda\) satisfies the additivity property if:

        \[
        \lambda_i^{(G^+)} = \lambda_i^{(G_1)} + \lambda_i^{(G_2)}
        \]

        Using the two games from part (c), demonstrate that the Shapley value has
        the additivity property.

        \hfill{[7]}
     
    \end{enumerate}

\newpage
\item

    \begin{enumerate}
        \item Define a \textbf{social welfare function}.

        \hfill{[2]}

        \item State \textbf{Arrow’s Impossibility Theorem}.
        Briefly discuss the implications of this theorem and ways in which the \textbf{Borda} or
        \textbf{Condorcet} methods respond to this impossibility.

        \hfill{[5]}

        \item Consider the following preference profile over the set of
        alternatives $X = \{A, B, C\}$:

        \[
        \begin{array}{c|ccc}
        \text{Number of voters} & \text{1st choice} & \text{2nd choice} & \text{3rd choice} \\
        \hline
        4 & A & B & C \\
        3 & B & C & A \\
        2 & C & A & B \\
        \end{array}
        \]

        \begin{enumerate}
            \item Construct the pairwise majority contests among the three
            alternatives.
            \item Determine if there is a \textbf{Condorcet winner}.
            \item Explain whether the collective preference relation is transitive.
        \end{enumerate}

        \hfill{[8]}

        \item Apply the \textbf{Borda count} method to the same profile.

        \begin{enumerate}
            \item Compute the Borda scores for each alternative.
            \item Identify the Borda winner.
            \item Does the Borda method select the same outcome as the
            Condorcet method?
        \end{enumerate}

        \hfill{[7]}

        \item Define what it means for a voting rule to satisfy the
        \textbf{Independence of Irrelevant Alternatives (IIA)} property.  
        Then, using the Borda count, give an example or explanation of how IIA
        may fail.

        \hfill{[3]}
    \end{enumerate}
\end{enumerate}

\makeatletter
\renewcommand{\@oddfoot}{\hfil \arabic{page}X \hfil}
\makeatother

\end{document}
