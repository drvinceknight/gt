\documentclass[12pt,a4paper,twoside]{article}
\usepackage{amsmath,amssymb}
\usepackage{graphicx}
\usepackage{float}
\usepackage{tikz}
\usetikzlibrary{calc}
\usepackage[english]{babel}
\usepackage[utf8]{inputenc}
\usetikzlibrary{shapes, arrows, positioning}


\usepackage{amsmath,amssymb,mdframed}            % AMS package gives better equation layouts


\usepackage{amsmath,amssymb,fancyhdr,graphicx,enumitem}
\setlength{\oddsidemargin}{-0.25in}     % set left margin
\setlength{\textwidth}{6.5in}           % set text width
\setlength{\topmargin}{-0.5in}          % controls layout at
\setlength{\headsep}{1cm}             % top of page
\setlength{\textheight}{9.0in}          % set text length

\usepackage{environ} % for custom environments
\newif\ifshowsolutions

\NewEnviron{solution}{%
  \ifshowsolutions
  \hrule
  \vspace{.5cm}
    \par\noindent\textbf{Solution:} 

    \BODY
  \vspace{.5cm}
    \hrule
  \fi
}

%\showsolutionstrue



\makeatletter
\renewcommand{\@oddhead}{\hfill MA3604/MOCK}  % sets header
\renewcommand{\@oddfoot}{\hfil \arabic{page} \hfil}    % sets page footer
\makeatother

\renewcommand{\labelenumi}{\arabic{enumi}} % Sets the first level of enumerate to be arabic (normal) numbers
\renewcommand{\labelenumii}{(\alph{enumii})} %Sets the second level of enumerate to be (a), (b), (c), .....
\renewcommand{\labelenumiii}{(\roman{enumiii})} % Sets the third level of enumerate to be (i), (ii), (iii), ....


\setlength{\marginparwidth}{0in}        % remove 'twoside' margins
\setlength{\marginparpush}{0in}         % remove 'twoside' margins
\setlength{\oddsidemargin}{-0.25in}     % set left margin
\setlength{\evensidemargin}{-0.25in}    % set left margin
\setlength{\textwidth}{6.5in}           % set text width
\setlength{\topmargin}{-0.5in}          % controls layout at
\setlength{\headsep}{0.5in}             % top of page
\setlength{\textheight}{9.0in}          % set text length

\pagestyle{fancy}                       % use following options
\fancyhead[R]{MA3604/MOCK}                % set header
\renewcommand{\headrulewidth}{0pt}      % replace ruled line at head
\setlength{\headheight}{-25pt}           % by whitespace

\fancyfoot[C]{-\,\arabic{page}\,-}      % set footer
\fancyfoot[RO]{Please turn over}        % add PTO to odd pages
% \fancyfoot[RO]{Trowch drosodd}        % add PTO to odd pages
\fancyfoot[RE]{}   


\begin{document}

\thispagestyle{plain}

\renewcommand{\familydefault}{\sfdefault}
\sffamily

\begin{minipage}[t]{25mm}
\vspace{0pt}
\includegraphics[width=21mm]{CUident_Black.pdf}
\end{minipage}%
\begin{minipage}[t]{130mm}
\vspace{0pt}
{\large \bf Do not turn this page over until instructed to do so by the Invigilation Supervisor}
% {\large \bf Peidiwch \^a throi'r dudalen hon nes i'r Arolygydd Goruchwylio roi caniat\^ad i chi}
\end{minipage}

\bigskip
{\bf Academic Year:} MOCK
% {\bf Blwyddyn Academaidd:} 2019/20

\medskip
{\bf Examination Period:} Autumn
% {\bf Cyfnod Arholiad:} Hydref

\medskip
{\bf Module Code:} MA3604
% {\bf Côd y Modiwl:} MAXXXX

\medskip
{\bf Examination Paper Title:} Game Theory
% {\bf Teitl y Papur Arholiad:} Enw'r modiwl

\medskip
{\bf Duration:} 2 hours
% {\bf Hyd:} 2/3 awr

\bigskip\bigskip
{\bf Please read the following information carefully:}
% {\bf Darllenwch y wybodaeth ganlynol yn ofalus:}

\bigskip\medskip
{\bf Structure of Examination Paper:}
% {\bf Strwythur y Papur Arholiad:}

\begin{itemize}[itemsep=-3pt]
\item There are $7$ pages including this page.  % specify the total number of pages (including the rubric page), written as a numeral, e.g. 7
% \item Mae $X$ tudalen, gan gynnwys hon.

\item There are {\bf 4} questions in total.     % specify the total number of questions, written in upper case letters, e.g. ELEVEN
% \item Mae {\bf X} o gwestiynau i gyd.

\item There are no appendices.                  % delete this or the previous item depending on whether or not there are appendices
% \item Nid oes unrhyw atodiadau.

\item The maximum mark for the examination paper is $75$ and the mark obtainable for a question or part of a question is shown in brackets alongside the question.
                                                % specify the total number of marks available (100 for 10-credit module, 150 for 20-credit module)
% \item $X$ yw'r marc uchaf ar gyfer y papur arholi a dangosir y marc sydd ar gael ar gyfer pob cwestiwn neu ran o gwestiwn mewn cromfachau ochr yn ochr â'r cwestiwn.
\end{itemize}

\bigskip
{\bf Instructions for completing the examination:}
% {\bf Cyfarwyddiadau ar gyfer llenwi'r arholiad:}

\begin{itemize}[itemsep=-3pt]
\item Complete the front cover of any answer books used.
% \item Llenwch dudalen flaen unrhyw lyfrynnau ateb a ddefnyddir.

\item This examination paper must be submitted to an Invigilator at the end of the examination. 
% \item Rhaid cyflwyno’r papur arholiad hwn i Oruchwyliwr ar ddiwedd yr arholiad.

\item Answer {\bf THREE} questions.
                                                % specify the number of questions to be answered from Section B, written in upper case letters, e.g TWO
% \item Atebwch {\bf BOB} cwestiwn o {\bf Adran A}, a {\bf THRI} chwestiwn o {\bf Adran B}.

\item Each question should be answered on a separate page.
% \item Dylid ateb bob cwestiwn ar dudalen wahanol.
\end{itemize}

\bigskip
{\bf You will be provided with / or allowed:}
% {\bf Byddwch yn cael / neu’n cael defnyddio:}

\begin{itemize}[itemsep=-3pt]
\item {\bf ONE} answer book.
% \item {\bf UN} llyfryn ateb.

\item The {\bf use of calculators} is {\bf permitted} in this examination.
% \item {\bf Ni chaniateir defnyddio cyfrifiannell} yn yr arholiad hwn.

\item The use of a translation dictionary between English or Welsh and another language, provided that it bears an appropriate departmental stamp, is permitted in this examination.
% \item Caniateir defnyddio geiriadur yn yr arholiad hwn i gyfieithu rhwng y Saesneg neu'r Gymraeg ac iaith arall, ar yr amod ei fod wedi cael stamp gan yr adran berthnasol.

\end{itemize}

\newpage

\renewcommand{\headrulewidth}{0pt}      % replace ruled line at head
\setlength{\headheight}{15pt}           % by whitespace

\renewcommand{\familydefault}{\rmdefault}
\rmfamily

\newpage
\begin{enumerate}

\renewcommand\labelenumi{\bfseries\theenumi.}

\item

    \begin{enumerate}
        \item Provide definitions for the following terms:
            \begin{itemize}
                \item Normal form game.

                    \begin{solution}
                        Book Work. \hfill{[1]}
                    \end{solution}

                \item Strictly dominated strategy.

                    \begin{solution}
                        Book Work. \hfill{[1]}
                    \end{solution}


                \item Weakly dominated strategy.

                    \begin{solution}
                        Book Work. \hfill{[1]}
                    \end{solution}

                \item Best response strategy.

                    \begin{solution}
                        Book Work. \hfill{[1]}
                    \end{solution}

                \item Nash equilibrium.

                    \begin{solution}
                        Book Work. \hfill{[1]}
                    \end{solution}


            \end{itemize}
                    ~\hfill{[5]}

        \item Consider the following Normal Form Game defined by:

            \[
                M_r=
\begin{pmatrix}
2 & 1 \\
1 & 3 \\
\end{pmatrix}
\qquad
            M_c=
\begin{pmatrix}
3 & 4 \\
3 & 2 \\
\end{pmatrix}
\]

            Where \(M_r\) corresponds to the payoffs of the row player and
            \(M_c\) corresponds to the payoffs of the column player.

            \begin{enumerate}
            \item Obtain all (if any) pure Nash equilibria (where each player chooses a single action with certainty).
            ~\hfill{[5]}

                \begin{solution}
                    Identifying the pure action best responses:

            \[
                M_r=
\begin{pmatrix}
    \underline{1} & 1 \\
    1 & \underline{3} \\
\end{pmatrix}
\qquad
M_c=
\begin{pmatrix}
    3 & \underline{4} \\
    \underline{3} & 2 \\
\end{pmatrix}
\]
                    \hfill{[4]}


                    There are no pure Nash equilibrium.
                    \hfill{[1]}
                \end{solution}


            \item Sketch the expected utilities to player 1 (the row player) of
                each action, assuming that the 2nd player (the column player) plays a strategy: 
                  $\sigma_2 = (y,1-y)$.
            ~\hfill{[4]}

                    \begin{solution}
                        We have:

                        % \[
                        %     u_1(r_1, (y, 1 - y)) = 7y
                        %     \qquad
                        %     u_1(r_2, (y, 1 - y)) = 2y+6-6y=6-4y
                        % \]
                        \hfill{[2]}

                        \begin{center}
% \begin{tikzpicture}[x=5cm]
%     % Axes
%     \draw[->] (0,0) -- (1,0) node[right] {$y$};
%     \draw[->] (0,0) -- (0,8) node[above] {$u_1$};
%
%     \draw[thick, blue, domain=0:1] plot(\x, {7*\x}) node[right] {$u_1(r_1, (y, 1 - y))$};
%
%     \draw[thick, red, domain=0:1] plot(\x, {6 - 4*\x}) node[right] {$u_1(r_2, (y, 1 - y))$};
%
%     % Intersection point
%     \filldraw[black] (6/11,42/11) circle (2pt) node[above right] {$(6/11,42/11)$};
% \end{tikzpicture}
                        \hfill{[2]}
                        \end{center}

                    \end{solution}


            \item Sketch the expected utilities to player 2 (the column player)
                of each action, assuming that the 1st player (the row player) plays a strategy: 
                 $\sigma_1 = (x,1-x)$.

            \hfill{[4]}

                    \begin{solution}
                        We have:

                        \[
                            u_2((x, 1 - x), c_1) = 3x
                            \qquad
                            u_2((x, 1 - x), c_2) = 2
                        \]
                        \hfill{[2]}

                        \begin{center}
% \begin{tikzpicture}[x=5cm]
%     % Axes
%     \draw[->] (0,0) -- (1,0) node[right] {$x$};
%     \draw[->] (0,0) -- (0,4) node[above] {$u_2$};
%
%     \draw[thick, blue, domain=0:1] plot(\x, {3*\x}) node[right] {$u_2((x, 1 - x), c_1)$};
%
%     \draw[thick, red, domain=0:1] plot(\x, {2}) node[right] {$u_2((x, 1 - x), c_2)$};
%
%     % Intersection point
%     \filldraw[black] (2/3,2) circle (2pt) node[above right] {$(2/3,2)$};
% \end{tikzpicture}
                        \hfill{[2]}
                        \end{center}

                    \end{solution}


            \item State and prove the best response condition theorem. Using this or
                otherwise
                obtain all Nash equilibria for the game. Confirm this using your
                    answers to question (ii) and (iii).

            ~\hfill{[7]}

            \begin{solution}
                State and prove theorem is bookwork. \hfill{[3]}

                % Using the theorem corresponds to the support enumeration
                % algorithm. We have found all NE with support size 1 (the pure
                % NE).
                % Now for support size 2.
                %
                % For $\sigma_2$:
                %
                % \[7y = 6 - 4 y\]
                % gives:
                %
                % \[y=6/11\]
                %
                % so $\sigma_1=(6/11, 5/11)$, as there are only two actions there is
                % no further support to explore.
                % \hfill{[1]}
                % 
                % For $\sigma_1$:
                %
                % \[3x=2\]
                % gives:
                % \[x=2/3\]
                %
                % so $\sigma_2=(2/3, 1/3)$
                % \hfill{[1]}
                %
                % From the plots from previous questions we have:
                %
                % \[
                %     \sigma_1^{*}
                %     \begin{cases}
                %             (0, 1)& y < 6/11\\
                %             (1, 0)& y > 6/11\\
                %             \text{indifferent}& y = 6/11\\
                %     \end{cases}
                %     \qquad
                %     \sigma_2^{*}
                %     \begin{cases}
                %             (0, 1)& x < 2/3\\
                %             (1, 0)& x > 1/3\\
                %             \text{indifferent}& x = 2/3\\
                %     \end{cases}
                % \]
                %
                %     \hfill{[1]}
                %
                % We see 3 pairs of best responses:
                %
                %     \[
                %         \{((1,0), (1, 0)), ((0,1), (0,1)), ((2/3, 1/3), (6/11, 5/11))\}
                %     \]
                %     \hfill{[1]}
                %
            \end{solution}

        \end{enumerate}
    \end{enumerate}

\newpage
\item

At a local \textbf{indie game convention}, two groups interact: \textbf{developers}
and \textbf{reviewers}.  

\begin{itemize}
  \item Each \textbf{developer} chooses whether to \textbf{Polish} their demo
  (\(P\)) or \textbf{Rush} to release it quickly (\(R\)).
  \item Each \textbf{reviewer} decides whether to \textbf{Highlight Indies}
  (\(H\)) or \textbf{Focus on Big Titles} (\(B\)).
\end{itemize}

If a developer polishes their demo and a reviewer highlights indies, both gain:
the developer earns attention, and the reviewer builds a reputation for
discovering quality games. If the reviewer focuses on big titles, a polished
indie receives less benefit, while rushed games may still get lucky exposure.
Reviewers, however, gain more visibility by covering popular titles — even if
the games themselves are weaker.

The interaction can be represented by two payoff matrices, one for each group.

\[
M_r =
\begin{pmatrix}
3 & 1 \\
2 & 2
\end{pmatrix},
\qquad
M_c =
\begin{pmatrix}
3 & 4 \\
5 & 1
\end{pmatrix},
\]

where \(M_r\) gives the payoffs to the \textbf{developers} (rows) and \(M_c\)
gives the payoffs to the \textbf{reviewers} (columns).

Let:
\begin{itemize}
  \item \(x_P\) and \(x_R = 1 - x_P\) denote the proportions of polishing and
  rushing developers;
  \item \(y_H\) and \(y_B = 1 - y_H\) denote the proportions of reviewers who
  highlight indies or focus on big titles.
\end{itemize}

\begin{enumerate}
    \item Compute the expected payoff of each strategy for both developers and
reviewers.

        \hfill{[2]}

\item Hence show that this interaction is modelled by the
\textbf{asymmetric replicator dynamics:}

\[
\begin{aligned}
    \frac{dx_P}{dt} &= - 2 x_{P}^{2} y_{H} + x_{P}^{2} + 2 x_{P} y_{H} - x_{P}\\
    \frac{dy_H}{dt} &= 5 x_{P} y_{H}^{2} - 5 x_{P} y_{H} - 4 y_{H}^{2} + 4 y_{H}
\end{aligned}
\]

        \hfill{[7]}

    \item Confirm that \(\{\sigma_r=(4/5, 1/5), \sigma_c=(1/2, 1/2))\}\) is a Nash equilibrium for
        the game defined by the payoff matrices.
        \hfill{[4]}

    \item Confirm that this Nash equilibrium corresponds to a stable population.

    \item For the asymmetric replicator dynamics equation for a pair of
        population vectors \((x, y)\), some \(\epsilon_x >0\), \(\epsilon_y>0\) and a pair of strategies \(\tilde x,
        \tilde y)\) a post entry population pair is given by:

        \[x_{\epsilon}=x+\epsilon_x(\tilde x - x)\qquad y_{\epsilon}=y+\epsilon_y(\tilde y - y)\]
        \hfill{[2]}

        Using this, propose a definition for an evolutionary stable strategy
        pair for the asymmetric replicator dynamics equation.\hfill{[6]}

    \item Discuss how you would use an appropriate numerical integration technique to
        explore the stability of points near the initial population. \textbf{You
        are not expected to carry out any calculations.}\hfill{[4]}

        
\end{enumerate}

\newpage
\item

        Consider the stage game defined by:

        \[
            M_r=
        \begin{pmatrix}
        2 & 5\\
        0 & 4\\
        \end{pmatrix}
        \qquad
            M_c=
        \begin{pmatrix}
        2 & 0\\
        5 & 4\\
        \end{pmatrix}
        \]

        The first action for both players will be referred to as `Cooperate'
            (\(C\)) and the second action will be referred to as `Defect' (\(D\)). \textbf{Players aim to minimise their payoffs.}

        Consider the following strategies for the infinitely repeated game with
            the above stage game.

            \begin{itemize}
                \item \(s_C\): Always cooperate;
                \item \(s_D\): Always defect;
                \item \(s_G\): Start by cooperating until your opponent defects at which point defect in all future stages.
            \end{itemize}

        Assume $A_1=A_2=\{s_C,s_D,s_G\}$.

    \begin{enumerate}

            \item Assuming a discounting factor of \(\delta\), obtain the
                utility to both players if the action pair \((s_C,s_C)\) is played.

            \hfill[2]

            \item Assuming a discounting factor of \(\delta\), obtain the
                utility to both players if the action pair \((s_D,s_D)\) is played.

            \hfill[2]

            \item For what values of \(\delta\) is \((s_G, s_G)\) a Nash equilibrium? \textbf{Recall that players aim to minimise their payoffs.}

            \hfill[5]

            \item Define the average payoff in an infinitely repeated game.

            \hfill[1]

            \item Plot the feasible average payoffs and the individually
                rational payoffs for the game. \textbf{Recall that players aim to minimise their payoffs.}

            \hfill[4]

            \item Prove the following theorem (\textbf{for games where players aim to minimise their payoffs}):

            ``Let \(u_1^*, u_2^*\) be a pair of Nash equilibrium payoffs for a stage game. For every individually rational pair \(v_1, v_2\) there exists \(\bar\delta\) such that for all \(1>\delta>\bar\delta>0\) there is a subgame perfect Nash equilibrium with payoffs \(v_1, v_2\).''

            \hfill[11]
    \end{enumerate}

\newpage
\item
    \begin{enumerate}
        \item Define a routing game \((G,r,c)\).

        \hfill{[2]}

        \item Define a Nash flow and using this definition obtain the Nash flow for the following game:

        \begin{center}
            \includegraphics[width=.5\textwidth]{images/img01.pdf}
        \end{center}

        \hfill{[3]}

        \item Define an optimal flow and using this definition obtain the optimal flow for the above game.

        \hfill{[3]}

        \item State the theorem connecting the following function \(\Phi\) to the Nash flow of a routing game:

        \[\Phi(f)=\sum_{e\in E}\int_{0}^{f_e}c_e(x)dx\]

        \hfill{[2]}

        \item Using the theorem from (d) confirm the Nash flow previously found in (b).

        \hfill{[2]}

        \item State the theorem connecting the marginal cost \(c^*(x)=\frac{d(xc(x))}{dx}\) to the optimal flow of a routing game.

        \hfill{[2]}

        \item Using the theorem from (f) confirm the optimal flow previously found in (c) .

        \hfill{[2]}

        \item The expected time spent in an \(M/M/1\) queue at steady state is given by:

        $$W_q=\frac{\lambda}{\mu(\mu-\lambda)}$$

        where \(\mu,\lambda\) are the mean service and inter arrival rates and \(\lambda < \mu\) respectively. Explain how a system with two \(M/M/1\) queues and players choosing which queue to join can be studied using the following routing game:

        \begin{center}
            \includegraphics[width=.5\textwidth]{images/img02.pdf}
        \end{center}

        \hfill{[2]}

        \item Obtain the Nash and Optimal flows for the game in (h) with \(\mu_1=4,\mu_2=3\) and \(\lambda=2\).

        You might find it useful to know that the equation:

        \[x^4-2x^3+x^2-420x+324=0\]

        has a single solution in the range $0\leq x<2$ given by $x\approx 0.7715$.

        \hfill{[7]}

    \end{enumerate}

\end{enumerate}

\makeatletter
\renewcommand{\@oddfoot}{\hfil \arabic{page}X \hfil}    % sets last page footer
\makeatother

\end{document}
