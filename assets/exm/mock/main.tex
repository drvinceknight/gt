\documentclass[12pt,a4paper]{article}
\usepackage{amsmath,amssymb}
\usepackage{graphicx}
\usepackage{float}
\usepackage{tikz}
\usetikzlibrary{shapes, arrows, positioning}


\usepackage{amsmath,amssymb,mdframed}            % AMS package gives better equation layouts
\setcounter{page}{5}                    % sets first page number to 2
\setlength{\oddsidemargin}{-0.25in}     % set left margin
\setlength{\textwidth}{6.5in}           % set text width
\setlength{\topmargin}{-0.5in}          % controls layout at
\setlength{\headsep}{0.5in}             % top of page
\setlength{\textheight}{9.0in}          % set text length



\makeatletter
\renewcommand{\@oddhead}{\hfill MA3600/Mock}  % sets header
\renewcommand{\@oddfoot}{\hfil \arabic{page} \hfil}    % sets page footer
\makeatother

\renewcommand{\labelenumi}{\arabic{enumi}} % Sets the first level of enumerate to be arabic (normal) numbers
\renewcommand{\labelenumii}{(\alph{enumii})} %Sets the second level of enumerate to be (a), (b), (c), .....
\renewcommand{\labelenumiii}{(\roman{enumiii})} % Sets the third level of enumerate to be (i), (ii), (iii), ....



\begin{document}
\null \vskip1cm
\begin{enumerate}

\renewcommand\labelenumi{\bfseries\theenumi.}

\item

    \begin{enumerate}
        \item Give the definition of Normal Form Game.
        ~\hfill{[2]}
        \item For the rest of this question, consider the Normal Form Game with
            the following matrix representation:
            $$A=
                \begin{pmatrix}
                    3 & 1\\
                    0 & 2
                \end{pmatrix}
              \qquad
              B=
                \begin{pmatrix}
                    2 & 1\\
                    0 & 3
                \end{pmatrix}
            $$
            Give the utilities to both players using the following strategy
            pairs:

            \begin{enumerate}
                \item $\sigma_r=(1, 0)\qquad\sigma_c=(0,1)$
                    ~\hfill{[1]}
                \item $\sigma_r=(1/2, 1/2)\qquad\sigma_c=(1/3,2/3)$
                    ~\hfill{[1]}
                \item $\sigma_r=(1/4, 3/4)\qquad\sigma_c=(0,1)$
                    ~\hfill{[1]}
            \end{enumerate}
        \item Give the definition for the Lemke-Howson algorithm.
            ~\hfill{[5]}
        \item Show that the vertices and their labels for the best response
            polytopes are given by:

            For the row player best response player \(\mathcal{P}\):

            \begin{itemize}
                \item $(0, 0)$ with labels: $\{0, 1\}$
                \item $(1/3, 0)$ with labels: $\{1, 2\}$
                \item $(0, 1/4)$ with labels: $\{0, 3\}$
                \item $(3/10, 1/10)$ with labels: $\{2, 3\}$
            \end{itemize}

            For the column player best response player \(\mathcal{Q}\):

            \begin{itemize}
                \item $(0, 0)$ with labels: $\{2, 3\}$
                \item $(1/4, 0)$ with labels: $\{0, 3\}$
                \item $(0, 1/3)$ with labels: $\{1, 2\}$
                \item $(1/10, 3/10)$ with labels: $\{0, 1\}$
            \end{itemize}

            ~\hfill{[4]}
        \item Draw the best response polytopes.
            ~\hfill{[2]}
        \item Use the plots to carry out the Lemke-Howson algorithm with all
            possible initial dropped labels.
            ~\hfill{[4]}
        \item Consider a modified Lemke-Howson algorithm that uses any pair of
            fully labeled vertices (and not necessarily (0, 0)).
            Use any pair of fully labeled vertices found in the previous question
            and find
            an initial dropped label that gives a different Nash equilibrium
            than the ones obtained in the previous question.
            ~\hfill{[1]}
        \item Give a sketch of a proof, including potential assumptions that in
            a non degenerate game the number of Nash equilibria is odd.
            ~\hfill{[4]}
    \end{enumerate}

\newpage
\item

    \begin{enumerate}

        \item Give the general definition of the Prisoner's Dilemma.
            ~\hfill{[2]}
        \item What values of $S, T$ give valid Prisoner's Dilemma games:
            \begin{enumerate}
                \item  \(A =
                         \begin{pmatrix}
                            3 & S\\
                            5 & 1
                         \end{pmatrix}
                         \qquad
                         B =
                         \begin{pmatrix}
                            3 & T\\
                            -1 & 1
                         \end{pmatrix}
                       \)
                ~\hfill{[4]}
                \item  \(A =
                         \begin{pmatrix}
                            2 & S\\
                            -2 & 1
                         \end{pmatrix}
                         \qquad
                         B =
                         \begin{pmatrix}
                            2 & -2\\
                            S & 1
                         \end{pmatrix}
                       \)
                ~\hfill{[4]}
            \end{enumerate}
        \item Consider the following reactive players:
            \[p = (3/5, 3/4)\qquad q = (1/2, 1/4)\]

            Obtain the Markov chain representation of a match between
                    these two players.
                    ~\hfill{[4]}
        \item State a theoretic result giving the utility of two general
            reactive players in a Prisoner's dilemma match (as a function of
            \((R, S, T, P)\)).
            ~\hfill{[5]}
        \item Consider a reactive player \(p=(x, x / 2)\) and an opponent
            \(q=(1/2, 1/4)\). Show that, for \((R, S, T, P)=(3, 0, 4, 1)\) the 
            utility to the player \(p\) is given by:
            \[
                u(x)=\frac{3x - 14}{\left(x - 8\right)}
            \]
            ~\hfill{[5]}
        \item Using the above, show that:
            \[
                \frac{du}{dx}=- \frac{10}{(x - 8) ^ 2}
            \]

            and use this to identify the optimal value of \(x\).

            ~\hfill{[4]}
    \end{enumerate}

\newpage
\item

    \begin{enumerate}
        \item For a matrix \(A=\begin{pmatrix}a&b\\c &d\end{pmatrix}\), obtain
              the following equation describing the corresponding evolutionary
              game:
              \[\frac{dx}{dt}=x(f-\phi)\]
              where \(f=Ax\) and \(\phi=fx\).
              ~\hfill{[2]}
        \item Define a mutated population.
              ~\hfill{[2]}
        \item Define an evolutionary stable strategy.
              ~\hfill{[2]}
        \item State and prove a theorem giving a general condition for an
            Evolutionary stable strategy.
              ~\hfill{[4]}
        \item Consider the following game
        \(A=\begin{pmatrix}1&4\\2 &1\end{pmatrix}\), obtain all evolutionary
            stable strategies.
              ~\hfill{[5]}
         \item Consider the accompanying 2008 paper entitled ``Studying the
             emergence of invasiveness in tumours using game theory'' by Basanta
             et al.
             \begin{enumerate}
                 \item Give a general summary of the paper.
                     ~\hfill{[3]}
                 \item There is a minor error in this paper in the game matrix, 
                     describe and suggest the fix.~\hfill{[2]}
                 \item How does the theorem in part 4 of this question relate to
                     the findings of the paper?~\hfill{[2]}
                 \item Suggest an alternative area of game theory that could
                     also be used.~\hfill{[3]}
             \end{enumerate}
    \end{enumerate}

\newpage
\item

    \begin{enumerate}
        \item Give the definition of a Moran process on a game.
            ~\hfill{[4]}
        \item State and prove a theorem giving the fixation probabilities for a
            general birth death process.
            ~\hfill{[6]}
        \item Consider the Moran process on the Prisoners Dilemma:
        \(A = \begin{pmatrix} 4 & 1\\ 6 & 2 \end{pmatrix}\)
            Use the above theorem to obtain the fixation probabilities for each
            strategy for \(N=5\).

            ~\hfill{[4]}
        \item Consider the following two strategies for the Prisoners Dilemma:
            \begin{itemize}
                \item Tit For Tat: start by cooperating and then repeat the
                    opponents previous message.
                \item Alternator: start by cooperating and then alternate
                    between defecting and cooperating.
            \end{itemize}
            Assuming a match lasting 5 turns show that the utility matrix
            between these two strategies corresponds to:

            \[
                \begin{pmatrix}
                    20 & 18\\
                    18 & 16
                \end{pmatrix}
            \]
                ~\hfill{[5]}
            \item Obtain the fixation probabilities \(x_1\) for each strategy
                for \(N=5\).
          ~\hfill{[6]}
    \end{enumerate}
\end{enumerate}


\makeatletter
\renewcommand{\@oddfoot}{\hfil \arabic{page}X \hfil}    % sets last page footer
\makeatother

\end{document}
